\documentclass[12pt, dvipdfmx]{jmaster}
%\documentclass[12pt, dvipdfmx]{jsotsuron}
\usepackage{newtxtext}
\usepackage{textcomp}
\usepackage[sc]{mathpazo}
\usepackage{amsmath,amssymb,arydshln, amsthm}
\usepackage[scaled]{helvet}
\usepackage{otf}
\usepackage[dvipdfmx]{graphicx}
\usepackage[dvipdfmx]{color}
\usepackage{tikz}
\usepackage[framemethod=tikz]{mdframed}
\usepackage{float}
\usepackage{subcaption}
\usepackage{verbatim}
\usepackage{ascmac}
%\usepackage{algorithm, algorithmic}
\usepackage{algpseudocode, algorithm}
\usepackage{listings,jlisting}
\usepackage{bm}
\usepackage{seqsplit}
\usepackage{url}

%追加
\usepackage{cite}
\usepackage{mathtools}
%\usepackage[]{algorithm2e}
%\usepackage{itembkbx}

\lstset{%
	language={C},
	basicstyle={\small},%
	identifierstyle={\small},%
	commentstyle={\small\itshape},%
	keywordstyle={\small\bfseries},%
	ndkeywordstyle={\small},%
	stringstyle={\small\ttfamily},
	frame={tb},
	breaklines=true,
	columns=[l]{fullflexible},%
	numbers=left,%
	xrightmargin=0zw,%
	xleftmargin=3zw,%
	numberstyle={\scriptsize},%
	stepnumber=1,
	numbersep=1zw,%
	lineskip=-0、5ex%
}

\theoremstyle{definition}
\newtheorem{theorem}{定理}
\newtheorem*{theorem*}{定理}
\newtheorem{definition}[theorem]{定義}
\newtheorem*{definition*}{定義}
\newtheorem{lemma}[theorem]{補題}
\newtheorem*{lemma*}{補題}
\newtheorem{Proof}{証明}
\newtheorem*{Proof*}{証明}

\newcommand{\argmax}{\mathop{\rm arg~max}\limits}
\newcommand{\argmin}{\mathop{\rm arg~min}\limits}
\newcommand{\name}[1]{\textbf{#1}}
% RequireとEnsureをInputとOutputにする
\renewcommand{\algorithmicrequire}{\textbf{Input:}}
\renewcommand{\algorithmicensure}{\textbf{Output:}}
\algnewcommand{\Initialize}[1]{%
  \State \textbf{Initialize:}
  \Statex \hspace*{\algorithmicindent}\parbox[t]{0.9\linewidth}{\raggedright #1}
  \Statex
}
%\algnewcommand{\IIf}[1]{\State\algorithmicif\ #1\ \algorithmicthen}
%\algnewcommand{\EndIIf}{\unskip\ \algorithmicend\ \algorithmicif}

\title{イジングマシンを用いた正確ロジスティック回帰\\
Exact Logistic Regression with Ising Machines}
\author{佐藤 史歩\\ Shiho Sato}

\date{February 2020}

\begin{document}
\maketitle
%$ $
%\newpage

\begin{abstract}
  統計検定は、生物学の研究において不可欠であり、
  通常は、漸近近似に基づく手法が用いられている。しかし、
  サンプル数が少ない場合には、誤差が大きくなるという問題が生じるが、
  全列挙に基づく正確検定手法は計算量が大きすぎるため、
  従来の計算機では実行が困難である。
   一方、量子アニーラーを初めとするイジングマシンは、近年急速に発展しており、多くの応用が期待されている。
  本論文では、量子アニーラを用いたサンプリングにより、
  Exact logistic regression(正確ロジスティック回帰)を行う手法を提案する。
  本手法では、適切なハミルトニアンを定義し、イジングマシンを用いて、
  その基底状態を求めることで、帰無分布のサンプリングを行う。
  古典計算機上でのシミュレーテッドアニーリング、及び、
  量子アニーラD-Wave 2000Qを用いて、提案法の実装を行った結果、
  全列挙に比較して、大幅に計算量を削減しながら、正確なP値の推定が可能
  であることがわかった。
  現状では、量子アニーラの規模が小さいため、小規模なデータにしか適用できないが、将来十分に量子計算機が発達すれば、GWASなどにも適用可能と考えられる。
\end{abstract}
\newpage
\tableofcontents
\newpage

\chapter{はじめに} 
\label{sec:intro}
\section{背景:統計解析とロジスティック回帰}
2003年にヒトゲノム計画が完了して以来、ゲノム配列の個人差と形質との関連についての研究が急速に進み、病気(特に単一遺伝子疾患)の原因遺伝子を探索する方法としてポジショナル・クローニングという手法が多用されてきた。ところが単一あるいは数個の遺伝子の異常のみで説明の付く病気はわずかで、大半が複数の遺伝子に少しずつ影響されることで発症する多因子疾患であり、従来の手法ではその解析は困難となった。そこで、これまでのような特定の狭い領域の解析に加えて、ゲノム全体を巨視的に見渡す必要性から、GWASが用いられるようになった。

GWASは遺伝的多様性を代表するSNPを位置マーカーとして用い、特定の病気と関連のあるSNPを見つけ出して、その近傍に存在すると推測される感受性遺伝子をリスト化していく。複数の感受性遺伝子を原因とする病気の場合、相当数の遺伝子の作用が累積することで、それなりの大きさの発症リスクをもたらすと考えられている。統計モデルとしてはロジスティック回帰を適用することが多い\cite{duverle2015privacy})。

ロジスティック回帰は一般化線形モデルのひとつであり、一般化線形モデルは一般線形モデルを拡張した概念である。一般線形モデル(Ordinary Linear Model)は応答変数が量的データであり、説明変数と応答変数の間の線形関係を仮定している。応答変数は正規分布に従い、その平均は説明変数の一次結合で表される。しかし現実世界には線形関係で表すことのできない事象もある。そこで応答変数が質的データであっても分析可能にしたのが一般化線形モデルである。一般化線形モデル(GLM:Generalized Linear Model)は一般線形モデルと異なり、指数型分布など正規分布以外に従う応答変数を予測する際にも使われる。説明変数との関係を推計ため、一般化線形モデルではlink関数を用いて変換した応答変数を用いる。Link関数とは応答変数を説明変数と線形関係になるよう変換する関数である。このような一般化線形モデルで最も使われている分析方法がロジスティック分析である。

ロジスティック回帰は、link関数にロジットを用いる。結果応答変数は必ず0-1の間となるため、確率を表すに適している。例えば、ある事件(event)が発生するか否かをロジスティック回帰モデルに当てはめると、その事件が発生する確率を予測することとなる。分析した結果、応答変数の値、すなわち確率が0.5より大きいとその事件が発生すると予測し、0.5を下回るとその事件が発生しないと予測する。ロジスティック分析における説明変数と応答変数との関係を示すとシグモイド曲線と呼ばれるS字形の曲線になる。

パラメータ推定の方法も一般線形モデルと一般化線形モデルでは異なる。一般線形モデルでは最小二乗法を用いてパラメータ推定を行う。最小二乗法は実測値と予測値の差である残差(residuals)の二乗和が最小になるパラメータを推定する方法である。これは、誤差は等分散正規分布となっているため、説明変数が1単位変化した際に応答変数がどのぐらい変化したのかを把握することが可能であることを利用している。一方ロジスティック回帰を含めた一般化線形モデルでは、最小二乗法を用いることができない。応答変数が質的データであるゆえ、説明変数との間に線形関係が存在せず、説明変数が1単位変化した際に被説明変数がどのぐらい変化したのかを把握することが難しいためである。従って、ロジスティック分析では最尤推定法 (Maximum Likelihood Estimation) という別の手段でパラメータ推定を行う。最尤推定はパラメータを変数、説明変数を定数とし、実測値が得られる最大の確率を知る手法である。

尤度比検定は最尤推定量を入れた尤度関数を用いて、帰無仮説の棄却を行う統計検定である。まず、帰無仮説に対する最尤推定量、対立仮説に対する最尤推定量それぞれを求める。次にそれぞれを尤度関数に代入し、その比をとる。ここで得られた比が、漸近的には自由度$(H_0 - H_1)$の$χ$二乗分布に従うことを利用したのが尤度比検定である。

ロジスティック回帰は最尤推定と尤度比検定を用いて統計解析を行うが、問題を抱えている。まず尤度比検定はサンプル数が大きいことを想定した検定であるため、サンプルが小さい場合に不正確になってしまうという問題がある。また、独立性について検定するため分割表を使用する際、疎であると最尤推定ができない場合も多い。偏りの大きいデータに対しても正確に検定を行うことが難しい。それらの問題を回避する方法の一つとしてExact logistic regression が提案されている。

\section{背景:Exact logistic regression}
Exact logistic regressionはCoxによって70年代に提案された統計的仮説検定で、共変量と応答変数ベクトルとの内積値を用いた条件付き分布を用いることで、検定したい説明変数に対する正確なP値を推定することが可能である\cite{cox1970analysis}。
P値は、統計的仮説検定において、帰無仮説の元で検定統計量がその値となる確率のことを指す。
P値が小さいほど、検定統計量がその値となることはあまり起こりえないことを意味する。一般にp値が5\%または1\%と事前に決められた有意水準$α(0<α<1)$以下となる場合、帰無仮説は偽として棄却される。
最尤推定を適用できない場合、すなわち通常のロジスティック回帰に対してサンプルサイズがとても小さい場合、データが疎あるいは同じ値をもつなど偏っている場合にexact logistic regressionは使われている\cite{10.2307/2288420, 10.2307/2289388, mehta1998exact, mehta1995exact}。またexact logistic regressionによって与えられた推定値は漸近的な結果に依存していない。
しかしこの方法は計算がNP完全であることが障壁となり、長年普及されてこなかった\cite{10.2307/2289388}。

\section{背景:量子アニーリング}
暗号解読のため第二次世界大戦の最中開発されたコンピュータは、真空管からトランジスタへと材料は変わったもののほぼ仕組みを変えずに現在まで開発されてきた。
性能の改善は、CPUなどの機能を作り出す集積回路に含まれるトランジスタの数を増やす(トランジスタの大きさを小さくする)ことで行われてきた。
しかしこの従来の方法による性能改善が限界を迎えている。トランジスタの大きさをこれ以上小さくしようとすると原子以下になってしまい、
今までの考え方が通用しなくなるからだ。これは一般にムーアの法則の限界として知られている。
またコンピュータにおける消費電力量も問題となっている。
2013年時点でコンピュータを中心とするITが消費する電力は年間1500TWhにのぼり、世界の発電量の1割にもなっている。
Google, Amazonといった巨大IT企業が冷却に必要な電力を減らすため比較的気温の低い地域にデータセンターを設置したり、再生エネルギーの導入に腐心していることから危機感が見て取れる。

そのような背景のもと、近年量子アニーリングが注目されている。
量子アニーリングはイジング模型の基底状態を量子効果を使って求める技術である。
量子は原子あるいはそれ以下の大きさであり、古典物理学ではなく量子力学に従う。
また消費電力も少ない。超伝導量子ビットを使った量子アニーリングは、CPUとメモリを合わせたチップの冷却用以外にほとんど電力を使わない上、極低温に冷却する部分はシステムが大きくなっても小さいままであるため、よりずっと大きなシステムになっても消費電力は基本的には変わらないからだ。
例えば量子アニーリングの消費電力は20kWである一方、スーパーコンピューター京は12MWである。能力が発揮される分野が異なるものの、600倍も節電が実現されている。
ちなみに、日本の一般家庭では年間400W消費するため、京は3万軒分の消費電力である。
さらに量子アニーリングは社会に存在する多くの問題の解決の一手となると期待されている。
量子アニーリングは組合せ最適化を高速に解くことができる手法であり、組合せ最適化問題は社会に多く潜んでいる。例えばルート最適化や配置問題などは量子アニーリングで高速に解くことのできる問題である。さらに近年は、量子アニーリングを利用した人工知能を支える機械学習アルゴリズムが様々開発されている。
よって、さらなる研究開発が進み、高速に大規模問題を解くことができるようになることが期待されている。
また、量子を用いるコンピューティングとしてゲート式という手法もあるが、量子アニーリングはゲート式と比較してシステムの安定性に優れているため、
搭載ビット数の多い実機が既に商用マシンとして開発され、比較的容易に利用できる利点がある\cite{nishimori}。

%ロジスティック回帰の検定に、Exact logistic regressionを適用したい。その際量子アニーリング技術を用いることで現実的な時間内での計算法を実現したい。
%jikken kekka

\section{主結果}
本研究では量子アニーリングを用いてExact logistic regressionを
現実的な時間内に行う手法を提案し、
実際に量子アニーラの実機を用いて実験を行なった。
Exact logistic regressionのP値計算におけるサンプリングを、
イジングモデルの基底状態サンプリングとして定式化した。
基底状態サンプリングは、古典計算機におけるシミュレーテッドアニーリング(SA)、
及び、量子アニーラ(QA)を用いて実装した。
各手法と比較検討した結果、QA、SAとも、全列挙と比べて、
大幅に計算量を削減しながら、正確なp値を得られていることがわかった。

\section{本論文の構成}
\ref{sec:propose}章では、以降の章で必要な用語と、
イジングマシンを用いたExact logistic regressionを説明する。 
\ref{sec:ex}章では、実データを用いて、各手法を用いたP値計算における、
計算時間と精度の比較を行う。
\ref{sec:cncls}章では、 本論文をまとめる。


\chapter{本論}
\label{sec:propose}
%本章では、イジングモデルを用いたExact logistic regressionについて述べる。
\section{量子アニーリング}
量子アニーリングは西森らによって1998年に開発された、量子効果を用いて組合せ最適化問題を解く手法のことである\cite{kadowaki1998quantum}。
重ね合わせの状態から、量子力学の核であるシュレディンガー方程式に従って自律的に変化していくことで、グローバル最適解を得ることができる。
この手法が対象とする組合せ最適化問題は、 エネルギーが
\begin{eqnarray}
	\label{quantum}
	H(\bm{\sigma}) = \sum_{i<j}J_{ij}\sigma_i \sigma_j + \sum_{i=1}^{N}h_i\sigma_i\\
	\sigma_i \in\{-1、+1\}\nonumber
\end{eqnarray}
で与えられる系(イジング模型)の基底状態(最低エネルギー状態)を求める問題とみなすことができる。
ここで$+1、 -1$の値をとる$\bm{\sigma}_i$はイジングスピンと呼ばれ、 $N$はスピン数を表す整数であり、 
問題サイズに対応する。このNが多くなると一般に解くことが難しくなり、基底状態を得るためにかかる時間が増加する。
また$J_{ij}$はスピン間の相互作用、 $h_i$は局所磁場を調整するパラメータである。
つまりこの問題は、 $\{J_{ij}、 h_i\}$が与えられた時、 エネルギー$H(\bm{\sigma})$を最小にする$\bm{\sigma}$を求める問題といえる。
量子アニーリングでははじめ、 量子効果(横磁場項)を大きくし、 各スピンの状態を不確定、 つまり$\pm1$を等確率で重ね合わせた状態に設定する。
そして、 量子効果を徐々に小さくすることで、 相互作用$J_{ij}$や局所磁場$h_i$の影響を強くしていく。
すると各スピンの状態が$\pm1$のどちらかに確定した状態が自律的に選ばれ、 それがイジング模型の基底状態となる。
つまり、 解きたい問題を$+1、 -1$の値をとる変数を持つ(\ref{quantum})式に落とし込み、 量子アニーリングを行うと、
その問題における目的関数$H(\bm{\sigma})$が最低エネルギーをとるような$\bm{\sigma}$が確率的に求まる。

カナダのD-Wave Systems Inc.は量子アニーリングの動作原理が実装されたハードウェアを世界で初めて開発し、
現在ではクラウド経由で利用できるD-Wave2000Qが提供されている\cite{dwave}。
現在D-Wave2000Qで扱うことのできる最大スピン数は2048個である。 
このD-Wave2000Qで採用されているグラフ構造(ワーキンググラフ)はキメラグラフと呼ばれ、
任意の$i,j(i<j)$の組に対して$J_{i,j}$を設定できるわけではない。
そのためには、 複数のスピンを仮想的な1つのスピンとみなすことで、 全結合グラフを再現する必要がある。
全結合グラフを作る際に扱えるスピン数は、 最大で64となる。
エネルギーを最小にする基底状態が複数ある場合には、量子アニーラを用いて、
基底状態のサンプリングを行うことができる~\cite{mandra2017exponentially}。
本研究では、この性質を生かし、統計検定への応用を行う。

\section{Exact logistic regression}

まず以下に示すデータを与える。
\begin{itemize}
\item $n$個の独立した標本データ
\item それぞれのデータに対する共変量を以下のように定義する。
begin{itemize}
		\item $x_{ij} \in\{0, 1\}, ~~ \mathbf{X}_j = [x_{1j}, \dots, x_{ij}, \dots , x_{nj}], ~~ \mathbf{X} = [\mathbf{X}_1, \dots, \mathbf{X}_j, \dots , \mathbf{X}_m]$
		\item 検定したい目的説明変数$X_k$(第i標本データにおいては$x_{ik}$)
	\end{itemize}
	\item それぞれのデータの応答変数は以下のように定義する
	\begin{itemize}
		\item $y_i \in\{0, 1\}, ~~ \mathbf{Y} = [y_1, \dots, y_i, \dots , y_n]$
	\end{itemize}
\end{itemize}

以上のデータが与えられた際、共変量に対するロジスティック回帰モデルは、以下のように表される。
\begin{eqnarray}
	\log \dfrac{\theta_i}{1- \theta_i} = \mathbf{x}^T_i \boldsymbol{\beta}
\end{eqnarray}
このとき $\mathbf{x}_i =(x_{i1}, \dots , x_{im})$は第$i$標本における$m$個の
共変量ベクトルであり、$~~ \boldsymbol{\beta} = (\beta_1, \dots , \beta_m)^{T}$は$m$個の回帰係数パラメータベクトルである。

Exact logistic regressionでは、パラメータを関心のある目的変数の回帰係数パラメータと、それ以外である局外パラメータに分ける。
そして局外パラメータの十分統計量の実現値で条件をつけて
目的パラメータの推測を行う。

Exact logistic regressionでは十分統計量として以下を使用する。
\begin{eqnarray}
	t_0= \sum_{i=1}^n y_i\\
	t_j= \sum_{i=1}^n x_{ij} y_i \nonumber
\end{eqnarray}

そして$t_k$以外に対して以下のように条件付けをし、条件を満たす$\mathbf{Y}$を全列挙する。$\hat{t}_j$は元々のデータセットにおける$\mathbf{Y}$に対して計算した十分統計量を表している。
\begin{eqnarray}
	\mathbf{Y} = \{y|t_0=\hat{t}_0,  t_j=\hat{t}_j, ~~ j \neq k\}
	\label{eq:sufstatistic}
\end{eqnarray}

式(\ref{eq:sufstatistic})を満たすすべてのYサンプルに対し$t_k$を計算し、その割合をΡ値とする。
\begin{eqnarray}
P(t_k = \hat{t}_k | t_0=\hat{t}_0,t_j=\hat{t}_j) = \dfrac{c( t_k=\hat{t}_k, t_j=\hat{t}_j)}{\sum_{u} c( t_k=u, t_j=\hat{t}_j)} ~~ ,j \neq k
\end{eqnarray}

$c(\alpha=\hat{\alpha}, \beta=\hat{\beta})$は$\alpha=\hat{\alpha}, \beta= \hat{\beta}$を満たすYの個数を表している。
以上がExact logistic regressionの手順である。

\section{イジングマシンを用いたExact logistic regression}
本研究では、条件を満たすYを得る過程にて全列挙を行わず、イジングマシンによる
サンプリングにより評価する。
イジングマシンにおいては、
QUBO(Quadratic unconstrained binary optimization)と呼ばれる二次式で表されるエネルギー関数(ハミルトニアン)を最小化する基底状態(最適解)を探索する。
基底状態が複数あるときは、文献\cite{sundar2019quantum}でも示されているように、その中からサンプリングを行うことができる。
Exact logistic regressionをイジングマシンを用いて行うには、
式(\ref{eq:sufstatistic})を満たす解が基底状態になるようなハミルトニアンを設計する必要がある。
ここでは以下のように設定した。
\begin{eqnarray}
	\begin{split}
H &= 2\sum_{p<q}y_py_q + (1-2\hat{t}_0)\sum y_p + \hat{t}_0^2 +\\
		& \quad \sum_{j \neq k}(2 \sum_{p < q}x_{pj} x_{qj} y_p y_q + (1-2 \hat{t}_j) \sum y_p x_{pj} + \hat{t}_j^2)
	\end{split}
\end{eqnarray}

量子アニーリングを用いたExact logistic regressionの手順は、以下の通りである。
まず解きたい問題をQUBO形式にし、量子アニーラへ入力する。
量子アニーラにおいて複数回の基底状態探索を繰り返すことで、複数の基底状態をサンプリングすることができる。しかしながらアニーリングマシンでは、得られる状態が基底状態ではなく、励起状態(すなわち,式(\ref{eq:sufstatistic})が満たされない解)が得られる場合がある。そのため、得られた解が望みの解になっているかは選別する必要がある。

Exact logistic regressionでは、検定したい特徴量と元のデータにおける応答変数との内積値$t_k$がどれほど有意かを知ることによる統計検定であるため、
得られたサンプルから、検定したい特徴量とサンプルとの内積値が
$\hat{t}_k$を超えるサンプルの割合$p$を計算する。

もし例えば有意水準を$α=0.05$と定めていたら、$p < 0.05$となるとき元のデータが有意である(対立仮説を棄却しない)とわかり、
そうでない場合は元のデータは有意でない(帰無仮説を棄却しない)とわかる。

以上が量子アニーリングを用いたExact logistic regressionである。
擬似コードをAlgorithm\ref{algo:qaelr}に示す。

{\scriptsize
	\begin{algorithm}
		\caption{量子アニーリングを用いたExact logistic regression}
		\label{algo:qaelr}
		\begin{algorithmic}
			\Require{入力データ$\mathbf{Y} =[y_1,  \dots , y_i,  \dots, y_n], \mathbf{X} = [\mathbf{X}_1, \dots , \mathbf{X}_j, \dots , \mathbf{X}_m]$}
			\Ensure{検定結果}
			\State $X_k \leftarrow$ 検定したい特徴量
			\State $\hat{t}_0 \leftarrow$ Yに関する十分統計量
			\State $\hat{t}_k \leftarrow$ $X_k$に関する十分統計量
			\State $\hat{t}_j \leftarrow$ $X_k$以外の共変量に関する十分統計量
			\State $H \leftarrow$ QUBO形式の目的関数
			
			\Procedure{\textsf{QA}}{$H$}
			\State {$H$を量子アニーラへ入力し、QAを行う}
			\State \textbf{return} $R(=numreads)$個のYベクトル$sampleY_1, \dots, sampleY_r, \dots, sampleY_R$
			\EndProcedure
			
			\Procedure{\textsf{SELECT\_VALID\_Y}}{$sampleY_1, \dots, sampleY_r, \dots, sampleY_R$}
			\State $validYlist= []$
			\For{$r = 1, \dots, R$}
			\If{$sampleY_r$に含まれる要素の和  $== \hat{t}_0$}
			\If{$sampleY_r$と$ \mathbf{X}_j$の内積 $== \hat{t}_j$}
			\State $append(validYlist, sampleY_r)$
			\EndIf
			\EndIf
			\EndFor
			\State \textbf{return} $validYlist$
			\EndProcedure
			
			\newpage
			\Procedure{\textsf{CALCULATE\_P}}{$validYlist$}
			\State $S  \leftarrow validYlist$に入っているYベクトルの数
			\State $sampleY_s \leftarrow validYlist$に含まれるYベクトル($s= 1, \dots , S$)
			\State $samplecount \leftarrow 0$
			\For{$s= 1, \dots , S$}
			\If{$sampleY_s$と$\mathbf{X}_k$の内積$== \hat{t}_k$}					
			\State $samplecount \leftarrow  samplecount + 1$
			\EndIf
			\EndFor
			\State $p \leftarrow samplecount/S$
			\State \textbf{return}$p$
			\EndProcedure
			
			\Procedure{\textsf{STATICAL\_HYPOTHESIS\_TESTING}}{$p$}
			\If{$p < \alpha$}
			\State 検定結果 $\leftarrow$ "帰無仮説を棄却する"
			\Else
			\State 検定結果 $\leftarrow$ "帰無仮説を棄却する"とする
			\EndIf
			\State \textbf{return}検定結果
			\EndProcedure
		\end{algorithmic} 
	\end{algorithm}
}


\chapter{実験}
\label{sec:ex}
本章では量子アニーリングを用いてExact logistic regressionを行い、 その性能を評価し、 その他サンプリング手法を用いたExact logistic regressionと比較する。

\section{実験環境}
実験用のデータには、骨肉腫データを用いた\cite{jaffe1983osteosarcoma}。
実験用のプログラムはPython3.7.0を用いて実装した。 
量子アニーラとしてD-Wave社のD-Wave2000Qを、クラウドサービスであるLeap(\url{https://cloud.dwavesys.com})を介して利用した。

\subsection{骨肉腫データ}
\label{sec:ost}
骨肉腫データは46サンプルを含む単変量解析結果のデータである\cite{jaffe1983osteosarcoma}。骨肉腫データのうち特徴量は3つ使用し、20サンプル、 25サンプル、30サンプルを46サンプルからランダムに選択したデータをそれぞれ5つ作成した。説明変数は多変量解析にて有意性が認められた共変量であり、1つ含まれている。
説明変数はリンパ球浸潤の有無(LI。有=1、無=0)を、それ以外の共変量は性別(SEX。女性=1、男性=0)、オステオイド(類骨)になんらかの病理が観察されるか否か(AOP。有=1、無=0)である。
応答変数も二値変数であり、無再発生存が3年以上だったか否か(DFI3。以上=1、未満=0)を示している。
表\ref{tb:ost}に詳細を記載する。
以降データサイズをbitと表現することとする。


\begin{table}[hbtp]
	\caption{骨肉腫データ}
	\label{tb:ost}
	\centering
	\begin{tabular}{ccccc}
		\hline
		LI & SEX & AOP & DFI3=1となるデータの割合\\
		\hline\hline
		0 & 0 & 0 & 3/3 (100\%)\\
		0 & 0 & 1 & 2/2 (100\%)\\
		0 & 1 & 0 & 4/4(100\%)\\
		0 & 1 & 1 & 1/1(100\%)\\
		1 & 0 & 0 & 5/5(100\%)\\
		1 & 0 & 1 & 3/5(100\%)\\
		1 & 1 & 0 & 5/9(100\%)\\
		1 & 1 & 1 & 6/17(100\%)\\
		\hline
	\end{tabular}
\end{table}


%autoを使用
\section{実験結果}
以下に実験結果を示す。

\subsection{結果1: チェーン強度とチェーンの破壊割合、チェーン強度とサンプル数}
\label{sec:chain}
D-Waveマシンはサンプリングを行う際、パラメータのひとつであるチェーン強度を調節することができる。
チェーンとは、1つの論理ビットを表すために複数の物理量子ビットをつなぐものを指し、
論理ビットをワーキンググラフ上に定義された物理量子ビットへと埋め込む際に必要に応じて用いる。
そこでD-waveを用いたサンプリングを実行するにあたり、事前にデータサイズごとに適切なチェーン強度を探索した。
適切さは、チェーンの破壊割合とサンプル数によって判断した。

20bitから40bitのデータをそれぞれ5つ用意し、チェーン強度を10, 15, 20, 25, 30 ,35と変化させサンプリングを実行した。
結果を図\ref{fig:broken}と、図\ref{fig:chain_num_sample}に示す。
各々y軸にデータを、x軸にチェーン強度を配置した。
図\ref{fig:broken}より、データサイズが大きいほど弱いチェーン強度ではチェーンが壊れやすいことがわかる。
また図\ref{fig:chain_num_sample}から、データサイズが小さいほどサンプル数が多くなるチェーン強度が小さいことがわかる。
以上の実験から20-40bitのデータに対する最適チェーン強度を決定すると同時に、同様に各チェーン強度候補に対し5回試行し46bitデータに対する最適チェーン強度を決定した。以下の表\ref{tb:chain_strength_datasize}にそれらの結果をまとめた。

\begin{figure}
	\begin{center}
		\includegraphics[width=130mm]{/Users/shihosato/Dropbox/sato_thesis_tex/figure/QA/broken_chain_proportion.png}
	\end{center}
	\caption{チェーン強度とチェーンの破壊割合}
	\label{fig:broken}
\end{figure}

\begin{figure}
	\begin{center}
		\includegraphics[width=130mm]{/Users/shihosato/Dropbox/sato_thesis_tex/figure/QA/chain_strength_valid_y_num.png}
	\end{center}
	\caption{チェーン強度とサンプル数}
	\label{fig:chain_num_sample}
\end{figure}

\begin{table}[hbtp]
	\caption{データサイズと最適チェーン強度}
	\label{tb:chain_strength_datasize}
	\centering
	\begin{tabular}{lcccccc}
		\hline
		&\multicolumn{6}{c}{データサイズ} \\
		& 20 & 25 & 30 & 35 & 40 & 46\\
		\hline\hline
		最適チェーン強度 & $15$ & $15$ & $15$ & $20$ & $20$ & $20$\\
		\hline
	\end{tabular}
\end{table}

%autoを使用
\subsection{結果2: 獲得サンプル数の比較}
\label{sec:time_compare}
結果\ref{sec:chain}より各データサイズに対するチェーン強度を決定したため、以降ではExact logistic regressionにおけるサンプリングを実行する。
サンプリング手法にはQAとしてD-waveを用いたサンプリングのほか、条件に合ったサンプルをすべて得る全列挙手法や、ランダムサンプリング、古典アニーリングであるシミュレーテッドアニーリング(SA)を実施し、各手法の比較検討を行った。

まず得られたサンプル数を手法、bitサイズごとに記録したものを表\ref{sample_num}に示す。
全列挙は計算に膨大な時間がかかるため30bitまでしか、結果を示すことができなかった。
ランダムサンプリングは40bitではサンプルを得ることができなかったため、35bitまでの結果を示す。

\begin{table}[hbtp]
	\caption{獲得サンプル数の比較}
	\label{tb:sample_num}
	\centering
	\begin{tabular}{lccccc}
		\hline
		&\multicolumn{5}{c}{データサイズ} \\
		手法& 20 & 25 & 30 & 35 & 40 \\
		\hline\hline
		全列挙 & $4.42\times 10^{3}$ & $2.79\times 10^{4}$ & $3.02\times 10^{5}$ & &  \\
		ランダムサンプリング & $4.46\times 10^{1}$ & $9.25$ & $4.50$ & $3.33$ & \\
		SA & $2.60\times 10^{3}$ & $6.46\times 10^{3}$ & $9.22\times 10^{3}$ & $9.62\times 10^{3}$ & $9.66\times 10^{3}$ \\
		QA & $2.91\times 10^{2}$ & $1.21\times 10^{2}$ & $8.24\times 10^{2}$ & $1.78\times 10^{1}$ & $8$ \\
		\hline
	\end{tabular}
\end{table}

\subsection{結果3: 計算時間の比較}
次に計算時間を計測した結果を図\ref{fig:total_time}に示す。
用いたデータは結果\ref{sec:chain}と同一で、各サイズに含まれる5つのデータの計測結果を、全列挙手法を除いて誤差範囲付き平均で表している。

40bitに至るまでサンプル獲得と時間計測が可能であったSAとQAを比較すると、QA のほうが計算時間が短いことがわかる。
さらに図\ref{fig:SA_QA_total_time}からより詳細に結果を観察すると、QAはデータサイズが大きくなるほどSAよりも短時間で計算ができることがわかる。
具体的な数値は表\ref{tb:sampling_time_SA_QA}に示す。

\begin{figure}
	\begin{center}
		\includegraphics[width=130mm]{/Users/shihosato/Dropbox/sato_thesis_tex/figure/auto_summary/_30bit_40bit_total_calculation_time_mean_std_log.png}
	\end{center}
	\caption{計算時間の比較}
	\label{fig:total_time}
\end{figure}

\begin{figure}
	\begin{center}
		\includegraphics[width=130mm]{/Users/shihosato/Dropbox/sato_thesis_tex/figure/auto_summary/_40bit_total_calculation_time_mean_std.png}
	\end{center}
	\caption{計算時間の比較 (SA, QA)}
	\label{fig:SA_QA_total_time}
\end{figure}


\begin{table}[hbtp]
	\caption{SAとQAにおけるサンプリング時間}
	\label{tb:sampling_time_SA_QA}
	\centering
	\begin{tabular}{lccccc}
		\hline
		& \multicolumn{5}{c}{データサイズ [sec]}\\
		手法& 20bit & 25bit & 30bit & 35bit & 40bit\\
		\hline\hline
		SA & $3.73$ & $5.08$ & $6.70$ & $8.36$ & $1.02\times 10^{1}$\\
		\hline
		QA & $2.40$ & $2.40$ & $2.40$ & $2.40$ & $2.40$\\
		\hline
		SA / QA & $1.55$ & $2.12$ & $2.53$ & $3.48$ & $4.26$\\
		\hline
	\end{tabular}
\end{table}

\subsection{結果4: p値の比較}
p値について得られた結果を図\ref{fig:p_value}に示す。
全列挙手法が30bitまでしか実施できなかったので、 30bitまでの結果を示す。
%図\ref{fig:p_value}から、SAは20bit, 25bit, 30bitにおいて、正確なp値を表すことが可能な全列挙の手法とほぼ同一の値を得ることができていることがわかる。
具体的な数値は\ref{tb:p_value}に示す。
ランダムサンプリングは20bit以外$\hat{t}_k \le t_k$となるサンプルを得られていないことがわかる。
表\ref{tb:enu_p_value}, 図\ref{fig:enu_p_value}には、全列挙と他の手法との平均の差を示した。


\begin{figure}
	\begin{center}
		\includegraphics[width=130mm]{/Users/shihosato/Dropbox/sato_thesis_tex/figure/auto_summary/_30bit_p_mean.png}
	\end{center}
	\caption{p値の比較}
	\label{fig:p_value}
\end{figure}

\begin{table}[hbtp]
\caption{p値の比較}
\label{tb:p_value}
\centering
\begin{tabular}{lccc}
  \hline
  & \multicolumn{3}{c}{データサイズ [sec]}\\手法& 20bit & 25bit & 30bit \\
		\hline\hline
		全列挙 & $0.292$ & $0.141$ & $0.148$ \\
		ランダムサンプリング & $0.40$ & $0.$ & $0.$ \\
		SA & $0.292$ & $0.138$ & $0.145$ \\
		QA & $0.294$ & $0.128$ & $0.166$ \\
		\hline
	\end{tabular}
\end{table}

\begin{table}[hbtp]
	\caption{全列挙とのp値の比較}
	\label{tb:enu_p_value}
	\centering
	\begin{tabular}{lccc}
		\hline
		& \multicolumn{3}{c}{データサイズ [sec]}\\
		手法& 20bit & 25bit & 30bit \\
		全列挙 & $0.292$ & $0.141$ & $0.148$ \\
		\hline\hline
		ランダムサンプリング & $+0.108$ & $-0.141$ & $-0.148$ \\
		SA & $0.$ & $-0.003$ & $-0.003$ \\
		QA & $+0.002$ & $-0.013$ & $+0.018$ \\
		\hline
	\end{tabular}
\end{table}

\begin{figure}
	\begin{center}
		\includegraphics[width=130mm]{/Users/shihosato/Dropbox/sato_thesis_tex/figure/auto_summary/_30bit_p_value_enu_diff.png}
	\end{center}
	\caption{全列挙とのp値の比較}
	\label{fig:enu_p_value}
\end{figure}

\subsection{結果5:46bitでの比較}
\ref{46bit_SA_QA}に46bitデータにて測定した結果を示す。

データの出典によると、今回説明変数として採用してる特徴量「リンパ球浸潤の有無」は単変量解析の結果p値0.006となるようだ\cite{jaffe1983osteosarcoma}。

\begin{table}[hbtp]
	\caption{46bitでの比較(SA, QA)}
	\label{tb:46bit_SA_QA}
	\centering
	\begin{tabular}{lcc}
		\hline
		& \multicolumn{2}{c}{手法}\\
		計測項目& SA& QA\\
		\hline\hline
		サンプル数& $9.46\times 10^{3}$ & $0$ \\
		\hline
		計算時間 & $1.27\times 10^{1}$ & $0.25$ \\
		\hline
		p値 & $0.04$ & $0.$ \\
		\hline
	\end{tabular}
\end{table}

\chapter{おわりに}
\label{sec:cncls}
本論文ではExact logistic regressionにおけるサンプルを得る工程を、
量子アニーラを用いて行う方法を提案した。
その具体的な方法として第\ref{sec:propose}章では、
量子アニーリングを用いたExact logistic regressionを提案した。
そして実際に、 量子アニーラの実機であるD-Wave2000Qを用いてサンプリングを行い、 同じくサンプリングへの適用が可能な手法
であるSAやランダムサンプリング、全列挙と比較した結果を示した。結果から、 SAと比較し得られるサンプル数は少ないものの、
全列挙の結果と近いp値を得られていることがわかった。
今後の課題として、 選択的推論への展開、他のイジングマシンへの展開などが挙げられる。

\chapter*{謝辞}
\label{sec:syazi}
終始熱心なご指導を頂いた津田 宏治教授,田村 亮講師に感謝の意を表します。
津田研究室の皆様から多くの刺激と示唆を得ることができました。感謝の意を表します。
本当にありがとうございました。



%%%%%%%%%%%%%%%%%%%% 参考文献 %%%%%%%%%%%%%%%%%%%%
\bibliographystyle{plain}
\bibliography{ref}

\end{document}
